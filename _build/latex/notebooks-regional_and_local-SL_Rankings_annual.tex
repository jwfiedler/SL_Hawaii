%% Generated by Sphinx.
\def\sphinxdocclass{jupyterBook}
\documentclass[letterpaper,10pt,english]{jupyterBook}
\ifdefined\pdfpxdimen
   \let\sphinxpxdimen\pdfpxdimen\else\newdimen\sphinxpxdimen
\fi \sphinxpxdimen=.75bp\relax
\ifdefined\pdfimageresolution
    \pdfimageresolution= \numexpr \dimexpr1in\relax/\sphinxpxdimen\relax
\fi
%% let collapsible pdf bookmarks panel have high depth per default
\PassOptionsToPackage{bookmarksdepth=5}{hyperref}
%% turn off hyperref patch of \index as sphinx.xdy xindy module takes care of
%% suitable \hyperpage mark-up, working around hyperref-xindy incompatibility
\PassOptionsToPackage{hyperindex=false}{hyperref}
%% memoir class requires extra handling
\makeatletter\@ifclassloaded{memoir}
{\ifdefined\memhyperindexfalse\memhyperindexfalse\fi}{}\makeatother


\PassOptionsToPackage{warn}{textcomp}

\catcode`^^^^00a0\active\protected\def^^^^00a0{\leavevmode\nobreak\ }
\usepackage{cmap}
\usepackage{fontspec}
\defaultfontfeatures[\rmfamily,\sffamily,\ttfamily]{}
\usepackage{amsmath,amssymb,amstext}
\usepackage{polyglossia}
\setmainlanguage{english}



\setmainfont{FreeSerif}[
  Extension      = .otf,
  UprightFont    = *,
  ItalicFont     = *Italic,
  BoldFont       = *Bold,
  BoldItalicFont = *BoldItalic
]
\setsansfont{FreeSans}[
  Extension      = .otf,
  UprightFont    = *,
  ItalicFont     = *Oblique,
  BoldFont       = *Bold,
  BoldItalicFont = *BoldOblique,
]
\setmonofont{FreeMono}[
  Extension      = .otf,
  UprightFont    = *,
  ItalicFont     = *Oblique,
  BoldFont       = *Bold,
  BoldItalicFont = *BoldOblique,
]



\usepackage[Bjarne]{fncychap}
\usepackage[,numfigreset=1,mathnumfig]{sphinx}

\fvset{fontsize=\small}
\usepackage{geometry}


% Include hyperref last.
\usepackage{hyperref}
% Fix anchor placement for figures with captions.
\usepackage{hypcap}% it must be loaded after hyperref.
% Set up styles of URL: it should be placed after hyperref.
\urlstyle{same}


\usepackage{sphinxmessages}



        % Start of preamble defined in sphinx-jupyterbook-latex %
         \usepackage[Latin,Greek]{ucharclasses}
        \usepackage{unicode-math}
        % fixing title of the toc
        \addto\captionsenglish{\renewcommand{\contentsname}{Contents}}
        \hypersetup{
            pdfencoding=auto,
            psdextra
        }
        % End of preamble defined in sphinx-jupyterbook-latex %
        

\title{Annual Sea Level Rankings}
\date{Jun 03, 2024}
\release{}
\author{Playground Coordinator}
\newcommand{\sphinxlogo}{\vbox{}}
\renewcommand{\releasename}{}
\makeindex
\begin{document}

\pagestyle{empty}
\sphinxmaketitle
\pagestyle{plain}
\sphinxtableofcontents
\pagestyle{normal}
\phantomsection\label{\detokenize{notebooks/regional_and_local/SL_Rankings_annual::doc}}


\begin{figure}[htbp]
\centering

\noindent\sphinxincludegraphics[scale=0.8]{{b827930c2a1f087092d409b27247bd84656d3504ee2990e12526b27172528ed7}.png}
\end{figure}

\sphinxAtStartPar
In this notebook, we’ll be creating a {\hyperref[\detokenize{notebooks/regional_and_local/SL_Rankings_annual:sl-rankings-results}]{\sphinxcrossref{\DUrole{std,std-ref}{table}}}}, a {\hyperref[\detokenize{notebooks/regional_and_local/SL_Rankings_annual:SL_rankings_map}]{\sphinxcrossref{\DUrole{xref,myst}{map}}}}, and a time series {\hyperref[\detokenize{notebooks/regional_and_local/SL_Rankings_annual:sl-rankings-timeseries}]{\sphinxcrossref{\DUrole{std,std-ref}{plot}}}} of sea level rankings at Hawaiian Islands tide gauge stations from 1993\sphinxhyphen{}2023.

\sphinxAtStartPar
Download Files:
\DUrole{xref,download,myst}{Map} |
\DUrole{xref,download,myst}{Time Series Plot} |
\DUrole{xref,download,myst}{Table}


\part{Setup}
\label{\detokenize{notebooks/regional_and_local/SL_Rankings_annual:setup}}
\sphinxAtStartPar
As with previous sections, we first need to import the necessary libraries, establish our input/output directories, and set up some basic plotting rules. We’ll do this by running another notebook called “setup” and “plotting functions,” and then we’ll set our data and output paths. If you have not run the datawrangling notebook, you will need to do this before running this notebook. Note that this notebook is also largely a repeat of the anomaly notebook.

\begin{sphinxShadowBox}
\sphinxstylesidebartitle{}

\begin{sphinxadmonition}{note}{Note:}
\sphinxAtStartPar
\sphinxstylestrong{TODO}
\begin{itemize}
\item {} 
\sphinxAtStartPar
Confirm numbers match

\item {} 
\sphinxAtStartPar
Figure captions

\item {} 
\sphinxAtStartPar
Figure labels

\item {} 
\sphinxAtStartPar
Text

\item {} 
\sphinxAtStartPar
Clean up code

\item {} 
\sphinxAtStartPar
add commentary

\end{itemize}
\end{sphinxadmonition}
\end{sphinxShadowBox}

\begin{sphinxuseclass}{cell}\begin{sphinxVerbatimInput}

\begin{sphinxuseclass}{cell_input}
\begin{sphinxVerbatim}[commandchars=\\\{\}]
\PYG{o}{\PYGZpc{}}\PYG{k}{run} ../setup.ipynb
\PYG{o}{\PYGZpc{}}\PYG{k}{run} ../plotting\PYGZus{}functions.ipynb
\PYG{n}{data\PYGZus{}dir} \PYG{o}{=} \PYG{n}{Path}\PYG{p}{(}\PYG{l+s+s1}{\PYGZsq{}}\PYG{l+s+s1}{../../data}\PYG{l+s+s1}{\PYGZsq{}} \PYG{p}{)}
\PYG{n}{output\PYGZus{}dir} \PYG{o}{=} \PYG{n}{Path}\PYG{p}{(}\PYG{l+s+s1}{\PYGZsq{}}\PYG{l+s+s1}{../../output}\PYG{l+s+s1}{\PYGZsq{}}\PYG{p}{)} 

\PYG{c+c1}{\PYGZsh{} We\PYGZsq{}re going to use plotly here, so we need to import it}
\PYG{k+kn}{import} \PYG{n+nn}{plotly}\PYG{n+nn}{.}\PYG{n+nn}{io} \PYG{k}{as} \PYG{n+nn}{pio}
\PYG{k+kn}{import} \PYG{n+nn}{plotly}\PYG{n+nn}{.}\PYG{n+nn}{express} \PYG{k}{as} \PYG{n+nn}{px}
\PYG{k+kn}{import} \PYG{n+nn}{plotly}\PYG{n+nn}{.}\PYG{n+nn}{offline} \PYG{k}{as} \PYG{n+nn}{py}
\PYG{k+kn}{import} \PYG{n+nn}{plotly}\PYG{n+nn}{.}\PYG{n+nn}{graph\PYGZus{}objects} \PYG{k}{as} \PYG{n+nn}{go}

\PYG{c+c1}{\PYGZsh{} check to make sure that data\PYGZus{}dir/rsl\PYGZus{}daily\PYGZus{}hawaii.nc exists, if not, make warning to run datawrangling notebook}
\PYG{k}{if} \PYG{o+ow}{not} \PYG{p}{(}\PYG{n}{data\PYGZus{}dir} \PYG{o}{/} \PYG{l+s+s1}{\PYGZsq{}}\PYG{l+s+s1}{rsl\PYGZus{}hawaii.nc}\PYG{l+s+s1}{\PYGZsq{}}\PYG{p}{)}\PYG{o}{.}\PYG{n}{exists}\PYG{p}{(}\PYG{p}{)}\PYG{p}{:}
    \PYG{n+nb}{print}\PYG{p}{(}\PYG{l+s+s1}{\PYGZsq{}}\PYG{l+s+s1}{rsl\PYGZus{}hawaii.nc not found in }\PYG{l+s+s1}{\PYGZsq{}}\PYG{o}{+} \PYG{n+nb}{str}\PYG{p}{(}\PYG{n}{data\PYGZus{}dir}\PYG{p}{)} \PYG{o}{+}  \PYG{l+s+s1}{\PYGZsq{}}\PYG{l+s+s1}{. Please run the data wrangling notebook first}\PYG{l+s+s1}{\PYGZsq{}}\PYG{p}{)}
\PYG{k}{else}\PYG{p}{:}
    \PYG{n+nb}{print}\PYG{p}{(}\PYG{l+s+s1}{\PYGZsq{}}\PYG{l+s+s1}{rsl\PYGZus{}hawaii.nc found in }\PYG{l+s+s1}{\PYGZsq{}}\PYG{o}{+} \PYG{n+nb}{str}\PYG{p}{(}\PYG{n}{data\PYGZus{}dir}\PYG{p}{)} \PYG{o}{+}  \PYG{l+s+s1}{\PYGZsq{}}\PYG{l+s+s1}{. Proceed.}\PYG{l+s+s1}{\PYGZsq{}}\PYG{p}{)}
\end{sphinxVerbatim}

\end{sphinxuseclass}\end{sphinxVerbatimInput}
\begin{sphinxVerbatimOutput}

\begin{sphinxuseclass}{cell_output}
\begin{sphinxVerbatim}[commandchars=\\\{\}]
rsl\PYGZus{}hawaii.nc found in ../../data. Proceed.
\end{sphinxVerbatim}

\end{sphinxuseclass}\end{sphinxVerbatimOutput}

\end{sphinxuseclass}

\chapter{Import data and Clean}
\label{\detokenize{notebooks/regional_and_local/SL_Rankings_annual:import-data-and-clean}}
\sphinxAtStartPar
Take only locations with data coverage more than 80\% of the time, and set all sea level measurements relative to local MHHW, rather than station datum.

\begin{sphinxuseclass}{cell}\begin{sphinxVerbatimInput}

\begin{sphinxuseclass}{cell_input}
\begin{sphinxVerbatim}[commandchars=\\\{\}]
\PYG{c+c1}{\PYGZsh{}import rsl\PYGZus{}daily}
\PYG{n}{rsl\PYGZus{}daily} \PYG{o}{=} \PYG{n}{xr}\PYG{o}{.}\PYG{n}{open\PYGZus{}dataset}\PYG{p}{(}\PYG{n}{data\PYGZus{}dir}\PYG{o}{/} \PYG{l+s+s1}{\PYGZsq{}}\PYG{l+s+s1}{rsl\PYGZus{}hawaii.nc}\PYG{l+s+s1}{\PYGZsq{}}\PYG{p}{)}
\PYG{n}{rsl\PYGZus{}daily} \PYG{o}{=} \PYG{n}{rsl\PYGZus{}daily}\PYG{o}{.}\PYG{n}{sel}\PYG{p}{(}\PYG{n}{time}\PYG{o}{=}\PYG{n+nb}{slice}\PYG{p}{(}\PYG{l+s+s1}{\PYGZsq{}}\PYG{l+s+s1}{1993}\PYG{l+s+s1}{\PYGZsq{}}\PYG{p}{,}\PYG{l+s+s1}{\PYGZsq{}}\PYG{l+s+s1}{2023}\PYG{l+s+s1}{\PYGZsq{}}\PYG{p}{)}\PYG{p}{)}

\PYG{n}{data\PYGZus{}coverage} \PYG{o}{=} \PYG{n}{rsl\PYGZus{}daily}\PYG{p}{[}\PYG{l+s+s1}{\PYGZsq{}}\PYG{l+s+s1}{sea\PYGZus{}level}\PYG{l+s+s1}{\PYGZsq{}}\PYG{p}{]}\PYG{o}{.}\PYG{n}{count}\PYG{p}{(}\PYG{n}{dim}\PYG{o}{=}\PYG{l+s+s1}{\PYGZsq{}}\PYG{l+s+s1}{time}\PYG{l+s+s1}{\PYGZsq{}}\PYG{p}{)}\PYG{o}{/}\PYG{n+nb}{len}\PYG{p}{(}\PYG{n}{rsl\PYGZus{}daily}\PYG{o}{.}\PYG{n}{time}\PYG{p}{)}

\PYG{n}{data\PYGZus{}coverage}
\PYG{c+c1}{\PYGZsh{}drop all locations with data\PYGZus{}coverage less than 80\PYGZpc{}}
\PYG{n}{rsl\PYGZus{}subset} \PYG{o}{=} \PYG{n}{rsl\PYGZus{}daily}\PYG{o}{.}\PYG{n}{where}\PYG{p}{(}\PYG{n}{data\PYGZus{}coverage}\PYG{o}{\PYGZgt{}}\PYG{l+m+mf}{0.80}\PYG{p}{,}\PYG{n}{drop}\PYG{o}{=}\PYG{k+kc}{True}\PYG{p}{)}

\PYG{c+c1}{\PYGZsh{} make sea level relative to MHHW, and convert to m}
\PYG{n}{rsl\PYGZus{}subset}\PYG{p}{[}\PYG{l+s+s1}{\PYGZsq{}}\PYG{l+s+s1}{sea\PYGZus{}level}\PYG{l+s+s1}{\PYGZsq{}}\PYG{p}{]} \PYG{o}{=} \PYG{n}{rsl\PYGZus{}subset}\PYG{p}{[}\PYG{l+s+s1}{\PYGZsq{}}\PYG{l+s+s1}{sea\PYGZus{}level}\PYG{l+s+s1}{\PYGZsq{}}\PYG{p}{]} \PYG{o}{\PYGZhy{}} \PYG{n}{rsl\PYGZus{}subset}\PYG{p}{[}\PYG{l+s+s1}{\PYGZsq{}}\PYG{l+s+s1}{MHHW}\PYG{l+s+s1}{\PYGZsq{}}\PYG{p}{]}
\PYG{n}{rsl\PYGZus{}subset}\PYG{p}{[}\PYG{l+s+s1}{\PYGZsq{}}\PYG{l+s+s1}{sea\PYGZus{}level}\PYG{l+s+s1}{\PYGZsq{}}\PYG{p}{]} \PYG{o}{=} \PYG{n}{rsl\PYGZus{}subset}\PYG{p}{[}\PYG{l+s+s1}{\PYGZsq{}}\PYG{l+s+s1}{sea\PYGZus{}level}\PYG{l+s+s1}{\PYGZsq{}}\PYG{p}{]}\PYG{o}{/}\PYG{l+m+mi}{1000}

\PYG{c+c1}{\PYGZsh{} rename variable long name to sea level relative to MHHW}
\PYG{n}{rsl\PYGZus{}subset}\PYG{p}{[}\PYG{l+s+s1}{\PYGZsq{}}\PYG{l+s+s1}{sea\PYGZus{}level}\PYG{l+s+s1}{\PYGZsq{}}\PYG{p}{]}\PYG{o}{.}\PYG{n}{attrs}\PYG{p}{[}\PYG{l+s+s1}{\PYGZsq{}}\PYG{l+s+s1}{long\PYGZus{}name}\PYG{l+s+s1}{\PYGZsq{}}\PYG{p}{]} \PYG{o}{=} \PYG{l+s+s1}{\PYGZsq{}}\PYG{l+s+s1}{sea level, MHHW}\PYG{l+s+s1}{\PYGZsq{}}
\PYG{n}{rsl\PYGZus{}subset}\PYG{p}{[}\PYG{l+s+s1}{\PYGZsq{}}\PYG{l+s+s1}{sea\PYGZus{}level}\PYG{l+s+s1}{\PYGZsq{}}\PYG{p}{]}\PYG{o}{.}\PYG{n}{attrs}\PYG{p}{[}\PYG{l+s+s1}{\PYGZsq{}}\PYG{l+s+s1}{units}\PYG{l+s+s1}{\PYGZsq{}}\PYG{p}{]} \PYG{o}{=} \PYG{l+s+s1}{\PYGZsq{}}\PYG{l+s+s1}{m}\PYG{l+s+s1}{\PYGZsq{}}
\end{sphinxVerbatim}

\end{sphinxuseclass}\end{sphinxVerbatimInput}

\end{sphinxuseclass}

\chapter{Resample to monthly}
\label{\detokenize{notebooks/regional_and_local/SL_Rankings_annual:resample-to-monthly}}
\sphinxAtStartPar
For background plotting purposes. Here we’ll extract the mean, min and max of the monthly MHHW sea level.

\begin{sphinxuseclass}{cell}\begin{sphinxVerbatimInput}

\begin{sphinxuseclass}{cell_input}
\begin{sphinxVerbatim}[commandchars=\\\{\}]
\PYG{c+c1}{\PYGZsh{} get min and max for each month by resampling}
\PYG{n}{rsl\PYGZus{}monthly\PYGZus{}min} \PYG{o}{=} \PYG{n}{rsl\PYGZus{}subset}\PYG{o}{.}\PYG{n}{resample}\PYG{p}{(}\PYG{n}{time}\PYG{o}{=}\PYG{l+s+s1}{\PYGZsq{}}\PYG{l+s+s1}{1ME}\PYG{l+s+s1}{\PYGZsq{}}\PYG{p}{)}\PYG{o}{.}\PYG{n}{min}\PYG{p}{(}\PYG{p}{)}
\PYG{n}{rsl\PYGZus{}monthly\PYGZus{}max} \PYG{o}{=} \PYG{n}{rsl\PYGZus{}subset}\PYG{o}{.}\PYG{n}{resample}\PYG{p}{(}\PYG{n}{time}\PYG{o}{=}\PYG{l+s+s1}{\PYGZsq{}}\PYG{l+s+s1}{1ME}\PYG{l+s+s1}{\PYGZsq{}}\PYG{p}{)}\PYG{o}{.}\PYG{n}{max}\PYG{p}{(}\PYG{p}{)}
\PYG{n}{rsl\PYGZus{}monthly\PYGZus{}mean} \PYG{o}{=} \PYG{n}{rsl\PYGZus{}subset}\PYG{o}{.}\PYG{n}{resample}\PYG{p}{(}\PYG{n}{time}\PYG{o}{=}\PYG{l+s+s1}{\PYGZsq{}}\PYG{l+s+s1}{1ME}\PYG{l+s+s1}{\PYGZsq{}}\PYG{p}{)}\PYG{o}{.}\PYG{n}{mean}\PYG{p}{(}\PYG{p}{)}
\end{sphinxVerbatim}

\end{sphinxuseclass}\end{sphinxVerbatimInput}

\end{sphinxuseclass}

\chapter{Define functions}
\label{\detokenize{notebooks/regional_and_local/SL_Rankings_annual:define-functions}}
\sphinxAtStartPar
The first will get us the top 10 high and low hourly sea level values for a given station. To ensure that we are tracking unique high and low water events, we ensure that the hourly maxima and minima are separated by at least 3 days.

\begin{sphinxuseclass}{cell}\begin{sphinxVerbatimInput}

\begin{sphinxuseclass}{cell_input}
\begin{sphinxVerbatim}[commandchars=\\\{\}]
\PYG{k}{def} \PYG{n+nf}{get\PYGZus{}top\PYGZus{}ten}\PYG{p}{(}\PYG{n}{rsl\PYGZus{}subset}\PYG{p}{,} \PYG{n}{rid}\PYG{p}{,} \PYG{n}{mode}\PYG{o}{=}\PYG{l+s+s1}{\PYGZsq{}}\PYG{l+s+s1}{max}\PYG{l+s+s1}{\PYGZsq{}}\PYG{p}{)}\PYG{p}{:}
    \PYG{c+c1}{\PYGZsh{} Convert data to a pandas Series}
    \PYG{n}{sea\PYGZus{}level\PYGZus{}series} \PYG{o}{=} \PYG{n}{rsl\PYGZus{}subset}\PYG{o}{.}\PYG{n}{sea\PYGZus{}level}\PYG{o}{.}\PYG{n}{isel}\PYG{p}{(}\PYG{n}{record\PYGZus{}id}\PYG{o}{=}\PYG{n}{rid}\PYG{p}{)}\PYG{o}{.}\PYG{n}{to\PYGZus{}series}\PYG{p}{(}\PYG{p}{)}


   \PYG{c+c1}{\PYGZsh{} Select top 100 values based on the mode}
    \PYG{k}{if} \PYG{n}{mode} \PYG{o}{==} \PYG{l+s+s1}{\PYGZsq{}}\PYG{l+s+s1}{max}\PYG{l+s+s1}{\PYGZsq{}}\PYG{p}{:}
        \PYG{n}{top\PYGZus{}values} \PYG{o}{=} \PYG{n}{sea\PYGZus{}level\PYGZus{}series}\PYG{o}{.}\PYG{n}{nlargest}\PYG{p}{(}\PYG{l+m+mi}{100}\PYG{p}{)}
    \PYG{k}{elif} \PYG{n}{mode} \PYG{o}{==} \PYG{l+s+s1}{\PYGZsq{}}\PYG{l+s+s1}{min}\PYG{l+s+s1}{\PYGZsq{}}\PYG{p}{:}
        \PYG{n}{top\PYGZus{}values} \PYG{o}{=} \PYG{n}{sea\PYGZus{}level\PYGZus{}series}\PYG{o}{.}\PYG{n}{nsmallest}\PYG{p}{(}\PYG{l+m+mi}{100}\PYG{p}{)}
    \PYG{k}{else}\PYG{p}{:}
        \PYG{k}{raise} \PYG{n+ne}{ValueError}\PYG{p}{(}\PYG{l+s+s1}{\PYGZsq{}}\PYG{l+s+s1}{mode must be either }\PYG{l+s+s1}{\PYGZdq{}}\PYG{l+s+s1}{max}\PYG{l+s+s1}{\PYGZdq{}}\PYG{l+s+s1}{ or }\PYG{l+s+s1}{\PYGZdq{}}\PYG{l+s+s1}{min}\PYG{l+s+s1}{\PYGZdq{}}\PYG{l+s+s1}{\PYGZsq{}}\PYG{p}{)}

    \PYG{c+c1}{\PYGZsh{} Filter to find unique events spaced by at least 3 days}
    \PYG{n}{filtered\PYGZus{}dates} \PYG{o}{=} \PYG{p}{[}\PYG{p}{]}
    \PYG{n}{top\PYGZus{}10\PYGZus{}values} \PYG{o}{=} \PYG{n}{pd}\PYG{o}{.}\PYG{n}{Series}\PYG{p}{(}\PYG{p}{)}

    \PYG{k}{for} \PYG{n}{date}\PYG{p}{,} \PYG{n}{value} \PYG{o+ow}{in} \PYG{n}{top\PYGZus{}values}\PYG{o}{.}\PYG{n}{items}\PYG{p}{(}\PYG{p}{)}\PYG{p}{:}
        \PYG{k}{if} \PYG{n+nb}{all}\PYG{p}{(}\PYG{n+nb}{abs}\PYG{p}{(}\PYG{p}{(}\PYG{n}{date} \PYG{o}{\PYGZhy{}} \PYG{n}{pd}\PYG{o}{.}\PYG{n}{to\PYGZus{}datetime}\PYG{p}{(}\PYG{n}{added\PYGZus{}date}\PYG{p}{)}\PYG{p}{)}\PYG{o}{.}\PYG{n}{days}\PYG{p}{)} \PYG{o}{\PYGZgt{}} \PYG{l+m+mi}{3} \PYG{k}{for} \PYG{n}{added\PYGZus{}date} \PYG{o+ow}{in} \PYG{n}{filtered\PYGZus{}dates}\PYG{p}{)}\PYG{p}{:}
            \PYG{n}{filtered\PYGZus{}dates}\PYG{o}{.}\PYG{n}{append}\PYG{p}{(}\PYG{n}{date}\PYG{p}{)}
            \PYG{n}{top\PYGZus{}10\PYGZus{}values}\PYG{p}{[}\PYG{n}{date}\PYG{p}{]} \PYG{o}{=} \PYG{n}{value}
        \PYG{k}{if} \PYG{n+nb}{len}\PYG{p}{(}\PYG{n}{filtered\PYGZus{}dates}\PYG{p}{)} \PYG{o}{==} \PYG{l+m+mi}{10}\PYG{p}{:}
            \PYG{k}{break}
    \PYG{n}{rank} \PYG{o}{=} \PYG{n}{np}\PYG{o}{.}\PYG{n}{arange}\PYG{p}{(}\PYG{l+m+mi}{1}\PYG{p}{,}\PYG{l+m+mi}{11}\PYG{p}{)}

    \PYG{n}{station\PYGZus{}name} \PYG{o}{=} \PYG{n+nb}{str}\PYG{p}{(}\PYG{n}{rsl\PYGZus{}subset}\PYG{p}{[}\PYG{l+s+s1}{\PYGZsq{}}\PYG{l+s+s1}{station\PYGZus{}name}\PYG{l+s+s1}{\PYGZsq{}}\PYG{p}{]}\PYG{o}{.}\PYG{n}{isel}\PYG{p}{(}\PYG{n}{record\PYGZus{}id}\PYG{o}{=}\PYG{n}{rid}\PYG{p}{)}\PYG{o}{.}\PYG{n}{values}\PYG{p}{)}
    \PYG{n}{record\PYGZus{}id} \PYG{o}{=} \PYG{n+nb}{str}\PYG{p}{(}\PYG{n}{rsl\PYGZus{}subset}\PYG{p}{[}\PYG{l+s+s1}{\PYGZsq{}}\PYG{l+s+s1}{record\PYGZus{}id}\PYG{l+s+s1}{\PYGZsq{}}\PYG{p}{]}\PYG{o}{.}\PYG{n}{isel}\PYG{p}{(}\PYG{n}{record\PYGZus{}id}\PYG{o}{=}\PYG{n}{rid}\PYG{p}{)}\PYG{o}{.}\PYG{n}{values}\PYG{p}{)}  

    \PYG{n}{top\PYGZus{}10\PYGZus{}values} \PYG{o}{=} \PYG{n}{pd}\PYG{o}{.}\PYG{n}{DataFrame}\PYG{p}{(}\PYG{p}{\PYGZob{}}\PYG{l+s+s1}{\PYGZsq{}}\PYG{l+s+s1}{rank}\PYG{l+s+s1}{\PYGZsq{}}\PYG{p}{:} \PYG{n}{rank}\PYG{p}{,} \PYG{l+s+s1}{\PYGZsq{}}\PYG{l+s+s1}{date}\PYG{l+s+s1}{\PYGZsq{}}\PYG{p}{:} \PYG{n}{top\PYGZus{}10\PYGZus{}values}\PYG{o}{.}\PYG{n}{index}\PYG{p}{,} \PYG{l+s+s1}{\PYGZsq{}}\PYG{l+s+s1}{sea level (m)}\PYG{l+s+s1}{\PYGZsq{}}\PYG{p}{:} \PYG{n}{top\PYGZus{}10\PYGZus{}values}\PYG{o}{.}\PYG{n}{values}\PYG{p}{\PYGZcb{}}\PYG{p}{)}  
    \PYG{n}{top\PYGZus{}10\PYGZus{}values}\PYG{p}{[}\PYG{l+s+s1}{\PYGZsq{}}\PYG{l+s+s1}{station\PYGZus{}name}\PYG{l+s+s1}{\PYGZsq{}}\PYG{p}{]} \PYG{o}{=} \PYG{n}{station\PYGZus{}name}
    \PYG{n}{top\PYGZus{}10\PYGZus{}values}\PYG{p}{[}\PYG{l+s+s1}{\PYGZsq{}}\PYG{l+s+s1}{record\PYGZus{}id}\PYG{l+s+s1}{\PYGZsq{}}\PYG{p}{]} \PYG{o}{=} \PYG{n}{record\PYGZus{}id}
    \PYG{n}{top\PYGZus{}10\PYGZus{}values}\PYG{p}{[}\PYG{l+s+s1}{\PYGZsq{}}\PYG{l+s+s1}{type}\PYG{l+s+s1}{\PYGZsq{}}\PYG{p}{]} \PYG{o}{=} \PYG{n}{mode}

    \PYG{c+c1}{\PYGZsh{}round the date to the nearest hour}
    \PYG{n}{top\PYGZus{}10\PYGZus{}values}\PYG{p}{[}\PYG{l+s+s1}{\PYGZsq{}}\PYG{l+s+s1}{date}\PYG{l+s+s1}{\PYGZsq{}}\PYG{p}{]} \PYG{o}{=} \PYG{n}{top\PYGZus{}10\PYGZus{}values}\PYG{p}{[}\PYG{l+s+s1}{\PYGZsq{}}\PYG{l+s+s1}{date}\PYG{l+s+s1}{\PYGZsq{}}\PYG{p}{]}\PYG{o}{.}\PYG{n}{dt}\PYG{o}{.}\PYG{n}{round}\PYG{p}{(}\PYG{l+s+s1}{\PYGZsq{}}\PYG{l+s+s1}{h}\PYG{l+s+s1}{\PYGZsq{}}\PYG{p}{)}

    \PYG{k}{return} \PYG{n}{top\PYGZus{}10\PYGZus{}values}
\end{sphinxVerbatim}

\end{sphinxuseclass}\end{sphinxVerbatimInput}

\end{sphinxuseclass}

\part{Make a Table}
\label{\detokenize{notebooks/regional_and_local/SL_Rankings_annual:make-a-table}}
\sphinxAtStartPar
We’ll use the function we defined above to make a table of the top 10 ranked high and low water events. We’ll add the ONI values to this table, which will come in handy later.

\begin{sphinxuseclass}{cell}\begin{sphinxVerbatimInput}

\begin{sphinxuseclass}{cell_input}
\begin{sphinxVerbatim}[commandchars=\\\{\}]
\PYG{k}{def} \PYG{n+nf}{get\PYGZus{}top\PYGZus{}10\PYGZus{}table}\PYG{p}{(}\PYG{n}{rsl\PYGZus{}subset}\PYG{p}{,}\PYG{n}{rid}\PYG{p}{)}\PYG{p}{:}
    \PYG{c+c1}{\PYGZsh{} make a table of the top 10 values, sorted by size and with date}
    \PYG{n}{top\PYGZus{}10\PYGZus{}values\PYGZus{}max} \PYG{o}{=} \PYG{n}{get\PYGZus{}top\PYGZus{}ten}\PYG{p}{(}\PYG{n}{rsl\PYGZus{}subset}\PYG{p}{,} \PYG{n}{rid}\PYG{p}{,} \PYG{n}{mode}\PYG{o}{=}\PYG{l+s+s1}{\PYGZsq{}}\PYG{l+s+s1}{max}\PYG{l+s+s1}{\PYGZsq{}}\PYG{p}{)}
    \PYG{n}{top\PYGZus{}10\PYGZus{}values\PYGZus{}min} \PYG{o}{=} \PYG{n}{get\PYGZus{}top\PYGZus{}ten}\PYG{p}{(}\PYG{n}{rsl\PYGZus{}subset}\PYG{p}{,} \PYG{n}{rid}\PYG{p}{,} \PYG{n}{mode}\PYG{o}{=}\PYG{l+s+s1}{\PYGZsq{}}\PYG{l+s+s1}{min}\PYG{l+s+s1}{\PYGZsq{}}\PYG{p}{)}

    \PYG{n}{top\PYGZus{}10\PYGZus{}table} \PYG{o}{=} \PYG{n}{pd}\PYG{o}{.}\PYG{n}{concat}\PYG{p}{(}\PYG{p}{[}\PYG{n}{top\PYGZus{}10\PYGZus{}values\PYGZus{}max}\PYG{p}{,}\PYG{n}{top\PYGZus{}10\PYGZus{}values\PYGZus{}min}\PYG{p}{]}\PYG{p}{)}

    \PYG{c+c1}{\PYGZsh{} cross reference the dates with the oni data to see if they are during an El Nino or La Nina event}
    \PYG{n}{oni} \PYG{o}{=} \PYG{n}{pd}\PYG{o}{.}\PYG{n}{read\PYGZus{}csv}\PYG{p}{(}\PYG{n}{data\PYGZus{}dir} \PYG{o}{/} \PYG{l+s+s1}{\PYGZsq{}}\PYG{l+s+s1}{oni.csv}\PYG{l+s+s1}{\PYGZsq{}}\PYG{p}{,} \PYG{n}{index\PYGZus{}col}\PYG{o}{=}\PYG{l+s+s1}{\PYGZsq{}}\PYG{l+s+s1}{Date}\PYG{l+s+s1}{\PYGZsq{}}\PYG{p}{,} \PYG{n}{parse\PYGZus{}dates}\PYG{o}{=}\PYG{k+kc}{True}\PYG{p}{)}

    \PYG{c+c1}{\PYGZsh{} add a new column to oni\PYGZus{}min called mode, where mode is either \PYGZsq{}El Nino\PYGZsq{}, \PYGZsq{}La Nina\PYGZsq{}, or \PYGZsq{}Neutral\PYGZsq{}}
    \PYG{n}{oni}\PYG{p}{[}\PYG{l+s+s1}{\PYGZsq{}}\PYG{l+s+s1}{ONI Mode}\PYG{l+s+s1}{\PYGZsq{}}\PYG{p}{]} \PYG{o}{=} \PYG{l+s+s1}{\PYGZsq{}}\PYG{l+s+s1}{Neutral}\PYG{l+s+s1}{\PYGZsq{}}
    \PYG{n}{oni}\PYG{o}{.}\PYG{n}{loc}\PYG{p}{[}\PYG{n}{oni}\PYG{p}{[}\PYG{l+s+s1}{\PYGZsq{}}\PYG{l+s+s1}{La Nina}\PYG{l+s+s1}{\PYGZsq{}}\PYG{p}{]} \PYG{o}{==}\PYG{k+kc}{True}\PYG{p}{,} \PYG{l+s+s1}{\PYGZsq{}}\PYG{l+s+s1}{ONI Mode}\PYG{l+s+s1}{\PYGZsq{}}\PYG{p}{]} \PYG{o}{=} \PYG{l+s+s1}{\PYGZsq{}}\PYG{l+s+s1}{La Nina}\PYG{l+s+s1}{\PYGZsq{}}
    \PYG{n}{oni}\PYG{o}{.}\PYG{n}{loc}\PYG{p}{[}\PYG{n}{oni}\PYG{p}{[}\PYG{l+s+s1}{\PYGZsq{}}\PYG{l+s+s1}{El Nino}\PYG{l+s+s1}{\PYGZsq{}}\PYG{p}{]} \PYG{o}{==}\PYG{k+kc}{True}\PYG{p}{,} \PYG{l+s+s1}{\PYGZsq{}}\PYG{l+s+s1}{ONI Mode}\PYG{l+s+s1}{\PYGZsq{}}\PYG{p}{]} \PYG{o}{=} \PYG{l+s+s1}{\PYGZsq{}}\PYG{l+s+s1}{El Nino}\PYG{l+s+s1}{\PYGZsq{}}

    \PYG{c+c1}{\PYGZsh{}drop the La Nina and El Nino columns}
    \PYG{n}{oni} \PYG{o}{=} \PYG{n}{oni}\PYG{o}{.}\PYG{n}{drop}\PYG{p}{(}\PYG{n}{columns}\PYG{o}{=}\PYG{p}{[}\PYG{l+s+s1}{\PYGZsq{}}\PYG{l+s+s1}{La Nina}\PYG{l+s+s1}{\PYGZsq{}}\PYG{p}{,} \PYG{l+s+s1}{\PYGZsq{}}\PYG{l+s+s1}{El Nino}\PYG{l+s+s1}{\PYGZsq{}}\PYG{p}{]}\PYG{p}{)}

    \PYG{c+c1}{\PYGZsh{}Extract ONI values for the corresponding dates of top\PYGZus{}10\PYGZus{}table}
    \PYG{n}{oni\PYGZus{}val} \PYG{o}{=} \PYG{n}{oni}\PYG{o}{.}\PYG{n}{reindex}\PYG{p}{(}\PYG{n}{top\PYGZus{}10\PYGZus{}table}\PYG{p}{[}\PYG{l+s+s1}{\PYGZsq{}}\PYG{l+s+s1}{date}\PYG{l+s+s1}{\PYGZsq{}}\PYG{p}{]}\PYG{p}{,} \PYG{n}{method}\PYG{o}{=}\PYG{l+s+s1}{\PYGZsq{}}\PYG{l+s+s1}{nearest}\PYG{l+s+s1}{\PYGZsq{}}\PYG{p}{)}

    \PYG{c+c1}{\PYGZsh{} add the oni values to the top\PYGZus{}10\PYGZus{}table}
    \PYG{n}{top\PYGZus{}10\PYGZus{}table} \PYG{o}{=} \PYG{n}{pd}\PYG{o}{.}\PYG{n}{merge}\PYG{p}{(}\PYG{n}{top\PYGZus{}10\PYGZus{}table}\PYG{p}{,} \PYG{n}{oni\PYGZus{}val}\PYG{p}{,} \PYG{n}{left\PYGZus{}on}\PYG{o}{=}\PYG{l+s+s1}{\PYGZsq{}}\PYG{l+s+s1}{date}\PYG{l+s+s1}{\PYGZsq{}}\PYG{p}{,} \PYG{n}{right\PYGZus{}index}\PYG{o}{=}\PYG{k+kc}{True}\PYG{p}{)}

    \PYG{k}{return} \PYG{n}{top\PYGZus{}10\PYGZus{}table}
\end{sphinxVerbatim}

\end{sphinxuseclass}\end{sphinxVerbatimInput}

\end{sphinxuseclass}
\sphinxAtStartPar
As a quick example, we can check the top of the table (the “head”) for the 4th record\_id, which is Kahului, Maui.

\begin{sphinxuseclass}{cell}\begin{sphinxVerbatimInput}

\begin{sphinxuseclass}{cell_input}
\begin{sphinxVerbatim}[commandchars=\\\{\}]
\PYG{n}{rid}\PYG{o}{=}\PYG{l+m+mi}{2}
\PYG{n}{top\PYGZus{}10\PYGZus{}table} \PYG{o}{=} \PYG{n}{get\PYGZus{}top\PYGZus{}10\PYGZus{}table}\PYG{p}{(}\PYG{n}{rsl\PYGZus{}subset}\PYG{p}{,}\PYG{n}{rid}\PYG{p}{)}
\PYG{n}{top\PYGZus{}10\PYGZus{}table}\PYG{o}{.}\PYG{n}{head}\PYG{p}{(}\PYG{p}{)}
\end{sphinxVerbatim}

\end{sphinxuseclass}\end{sphinxVerbatimInput}
\begin{sphinxVerbatimOutput}

\begin{sphinxuseclass}{cell_output}
\begin{sphinxVerbatim}[commandchars=\\\{\}]
   rank                date  sea level (m)      station\PYGZus{}name record\PYGZus{}id type  \PYGZbs{}
0     1 2020\PYGZhy{}12\PYGZhy{}15 15:00:00          0.458  Honolulu, Hawaii       570  max   
1     2 2019\PYGZhy{}12\PYGZhy{}25 14:00:00          0.447  Honolulu, Hawaii       570  max   
2     3 2017\PYGZhy{}08\PYGZhy{}21 01:00:00          0.440  Honolulu, Hawaii       570  max   
3     4 2021\PYGZhy{}12\PYGZhy{}05 15:00:00          0.428  Honolulu, Hawaii       570  max   
4     5 2020\PYGZhy{}07\PYGZhy{}20 02:00:00          0.423  Honolulu, Hawaii       570  max   

    ONI ONI Mode  
0 \PYGZhy{}1.19  La Nina  
1  0.55  Neutral  
2 \PYGZhy{}0.11  Neutral  
3 \PYGZhy{}0.98  La Nina  
4 \PYGZhy{}0.41  Neutral  
\end{sphinxVerbatim}

\end{sphinxuseclass}\end{sphinxVerbatimOutput}

\end{sphinxuseclass}

\chapter{Display by rank}
\label{\detokenize{notebooks/regional_and_local/SL_Rankings_annual:display-by-rank}}
\sphinxAtStartPar
Now we’ll start to make the table a little more presentable by labeling the columns properly, and organizing events by rank.

\begin{sphinxuseclass}{cell}\begin{sphinxVerbatimInput}

\begin{sphinxuseclass}{cell_input}
\begin{sphinxVerbatim}[commandchars=\\\{\}]
\PYG{c+c1}{\PYGZsh{} subdivide the data into two columns for max and min}
\PYG{c+c1}{\PYGZsh{} Filter out the \PYGZsq{}max\PYGZsq{} and \PYGZsq{}min\PYGZsq{} types into separate DataFrames}
\PYG{n}{max\PYGZus{}top\PYGZus{}10} \PYG{o}{=} \PYG{n}{top\PYGZus{}10\PYGZus{}table}\PYG{p}{[}\PYG{n}{top\PYGZus{}10\PYGZus{}table}\PYG{p}{[}\PYG{l+s+s1}{\PYGZsq{}}\PYG{l+s+s1}{type}\PYG{l+s+s1}{\PYGZsq{}}\PYG{p}{]} \PYG{o}{==} \PYG{l+s+s1}{\PYGZsq{}}\PYG{l+s+s1}{max}\PYG{l+s+s1}{\PYGZsq{}}\PYG{p}{]}\PYG{o}{.}\PYG{n}{reset\PYGZus{}index}\PYG{p}{(}\PYG{n}{drop}\PYG{o}{=}\PYG{k+kc}{True}\PYG{p}{)}
\PYG{n}{min\PYGZus{}top\PYGZus{}10} \PYG{o}{=} \PYG{n}{top\PYGZus{}10\PYGZus{}table}\PYG{p}{[}\PYG{n}{top\PYGZus{}10\PYGZus{}table}\PYG{p}{[}\PYG{l+s+s1}{\PYGZsq{}}\PYG{l+s+s1}{type}\PYG{l+s+s1}{\PYGZsq{}}\PYG{p}{]} \PYG{o}{==} \PYG{l+s+s1}{\PYGZsq{}}\PYG{l+s+s1}{min}\PYG{l+s+s1}{\PYGZsq{}}\PYG{p}{]}\PYG{o}{.}\PYG{n}{reset\PYGZus{}index}\PYG{p}{(}\PYG{n}{drop}\PYG{o}{=}\PYG{k+kc}{True}\PYG{p}{)}

\PYG{c+c1}{\PYGZsh{} make table of highest and lowest values by rank, with columns: rank, highest sea level (m), date, lowest sea level (m), date}
\PYG{n}{top\PYGZus{}10\PYGZus{}display} \PYG{o}{=} \PYG{n}{pd}\PYG{o}{.}\PYG{n}{DataFrame}\PYG{p}{(}\PYG{p}{\PYGZob{}}\PYG{l+s+s1}{\PYGZsq{}}\PYG{l+s+s1}{Rank}\PYG{l+s+s1}{\PYGZsq{}}\PYG{p}{:}\PYG{n}{max\PYGZus{}top\PYGZus{}10}\PYG{p}{[}\PYG{l+s+s1}{\PYGZsq{}}\PYG{l+s+s1}{rank}\PYG{l+s+s1}{\PYGZsq{}}\PYG{p}{]}\PYG{p}{,}
                             \PYG{l+s+s1}{\PYGZsq{}}\PYG{l+s+s1}{Highest}\PYG{l+s+s1}{\PYGZsq{}}\PYG{p}{:}\PYG{n}{max\PYGZus{}top\PYGZus{}10}\PYG{p}{[}\PYG{l+s+s1}{\PYGZsq{}}\PYG{l+s+s1}{sea level (m)}\PYG{l+s+s1}{\PYGZsq{}}\PYG{p}{]}\PYG{p}{,}
                             \PYG{l+s+s1}{\PYGZsq{}}\PYG{l+s+s1}{Highest Date}\PYG{l+s+s1}{\PYGZsq{}}\PYG{p}{:}\PYG{n}{max\PYGZus{}top\PYGZus{}10}\PYG{p}{[}\PYG{l+s+s1}{\PYGZsq{}}\PYG{l+s+s1}{date}\PYG{l+s+s1}{\PYGZsq{}}\PYG{p}{]}\PYG{o}{.}\PYG{n}{dt}\PYG{o}{.}\PYG{n}{strftime}\PYG{p}{(}\PYG{l+s+s1}{\PYGZsq{}}\PYG{l+s+s1}{\PYGZpc{}}\PYG{l+s+s1}{Y\PYGZhy{}}\PYG{l+s+s1}{\PYGZpc{}}\PYG{l+s+s1}{m\PYGZhy{}}\PYG{l+s+si}{\PYGZpc{}d}\PYG{l+s+s1}{ }\PYG{l+s+s1}{\PYGZpc{}}\PYG{l+s+s1}{H:}\PYG{l+s+s1}{\PYGZpc{}}\PYG{l+s+s1}{M}\PYG{l+s+s1}{\PYGZsq{}}\PYG{p}{)}\PYG{p}{,}
                             \PYG{l+s+s1}{\PYGZsq{}}\PYG{l+s+s1}{Highest ONI Mode}\PYG{l+s+s1}{\PYGZsq{}}\PYG{p}{:}\PYG{n}{max\PYGZus{}top\PYGZus{}10}\PYG{p}{[}\PYG{l+s+s1}{\PYGZsq{}}\PYG{l+s+s1}{ONI Mode}\PYG{l+s+s1}{\PYGZsq{}}\PYG{p}{]}\PYG{p}{,}
                             \PYG{l+s+s1}{\PYGZsq{}}\PYG{l+s+s1}{Lowest}\PYG{l+s+s1}{\PYGZsq{}}\PYG{p}{:}\PYG{n}{min\PYGZus{}top\PYGZus{}10}\PYG{p}{[}\PYG{l+s+s1}{\PYGZsq{}}\PYG{l+s+s1}{sea level (m)}\PYG{l+s+s1}{\PYGZsq{}}\PYG{p}{]}\PYG{p}{,}
                             \PYG{l+s+s1}{\PYGZsq{}}\PYG{l+s+s1}{Lowest Date}\PYG{l+s+s1}{\PYGZsq{}}\PYG{p}{:}\PYG{n}{min\PYGZus{}top\PYGZus{}10}\PYG{p}{[}\PYG{l+s+s1}{\PYGZsq{}}\PYG{l+s+s1}{date}\PYG{l+s+s1}{\PYGZsq{}}\PYG{p}{]}\PYG{o}{.}\PYG{n}{dt}\PYG{o}{.}\PYG{n}{strftime}\PYG{p}{(}\PYG{l+s+s1}{\PYGZsq{}}\PYG{l+s+s1}{\PYGZpc{}}\PYG{l+s+s1}{Y\PYGZhy{}}\PYG{l+s+s1}{\PYGZpc{}}\PYG{l+s+s1}{m\PYGZhy{}}\PYG{l+s+si}{\PYGZpc{}d}\PYG{l+s+s1}{ }\PYG{l+s+s1}{\PYGZpc{}}\PYG{l+s+s1}{H:}\PYG{l+s+s1}{\PYGZpc{}}\PYG{l+s+s1}{M}\PYG{l+s+s1}{\PYGZsq{}}\PYG{p}{)}\PYG{p}{,}
                             \PYG{l+s+s1}{\PYGZsq{}}\PYG{l+s+s1}{Lowest ONI Mode}\PYG{l+s+s1}{\PYGZsq{}}\PYG{p}{:}\PYG{n}{min\PYGZus{}top\PYGZus{}10}\PYG{p}{[}\PYG{l+s+s1}{\PYGZsq{}}\PYG{l+s+s1}{ONI Mode}\PYG{l+s+s1}{\PYGZsq{}}\PYG{p}{]}\PYG{p}{,}
                             \PYG{l+s+s1}{\PYGZsq{}}\PYG{l+s+s1}{Zone}\PYG{l+s+s1}{\PYGZsq{}}\PYG{p}{:} \PYG{l+s+s1}{\PYGZsq{}}\PYG{l+s+s1}{GMT}\PYG{l+s+s1}{\PYGZsq{}}\PYG{p}{,}
                             \PYG{l+s+s1}{\PYGZsq{}}\PYG{l+s+s1}{ONI max}\PYG{l+s+s1}{\PYGZsq{}}\PYG{p}{:}\PYG{n}{max\PYGZus{}top\PYGZus{}10}\PYG{p}{[}\PYG{l+s+s1}{\PYGZsq{}}\PYG{l+s+s1}{ONI}\PYG{l+s+s1}{\PYGZsq{}}\PYG{p}{]}\PYG{p}{,}
                             \PYG{l+s+s1}{\PYGZsq{}}\PYG{l+s+s1}{ONI min}\PYG{l+s+s1}{\PYGZsq{}}\PYG{p}{:}\PYG{n}{min\PYGZus{}top\PYGZus{}10}\PYG{p}{[}\PYG{l+s+s1}{\PYGZsq{}}\PYG{l+s+s1}{ONI}\PYG{l+s+s1}{\PYGZsq{}}\PYG{p}{]}\PYG{p}{\PYGZcb{}}\PYG{p}{)}

\PYG{n}{top\PYGZus{}10\PYGZus{}display}
\end{sphinxVerbatim}

\end{sphinxuseclass}\end{sphinxVerbatimInput}
\begin{sphinxVerbatimOutput}

\begin{sphinxuseclass}{cell_output}
\begin{sphinxVerbatim}[commandchars=\\\{\}]
   Rank  Highest      Highest Date Highest ONI Mode  Lowest       Lowest Date  \PYGZbs{}
0     1    0.458  2020\PYGZhy{}12\PYGZhy{}15 15:00          La Nina  \PYGZhy{}0.831  2011\PYGZhy{}03\PYGZhy{}11 16:00   
1     2    0.447  2019\PYGZhy{}12\PYGZhy{}25 14:00          Neutral  \PYGZhy{}0.768  2011\PYGZhy{}03\PYGZhy{}18 20:00   
2     3    0.440  2017\PYGZhy{}08\PYGZhy{}21 01:00          Neutral  \PYGZhy{}0.765  1994\PYGZhy{}03\PYGZhy{}29 21:00   
3     4    0.428  2021\PYGZhy{}12\PYGZhy{}05 15:00          La Nina  \PYGZhy{}0.748  2010\PYGZhy{}02\PYGZhy{}28 20:00   
4     5    0.423  2020\PYGZhy{}07\PYGZhy{}20 02:00          Neutral  \PYGZhy{}0.745  2009\PYGZhy{}05\PYGZhy{}24 19:00   
5     6    0.415  2020\PYGZhy{}11\PYGZhy{}15 14:00          La Nina  \PYGZhy{}0.744  2002\PYGZhy{}02\PYGZhy{}26 07:00   
6     7    0.411  2020\PYGZhy{}10\PYGZhy{}19 16:00          La Nina  \PYGZhy{}0.743  2010\PYGZhy{}04\PYGZhy{}27 19:00   
7     8    0.410  2017\PYGZhy{}06\PYGZhy{}24 02:00          Neutral  \PYGZhy{}0.739  2011\PYGZhy{}12\PYGZhy{}25 07:00   
8     9    0.406  2020\PYGZhy{}07\PYGZhy{}05 02:00          Neutral  \PYGZhy{}0.739  2013\PYGZhy{}04\PYGZhy{}25 19:00   
9    10    0.401  2019\PYGZhy{}08\PYGZhy{}02 03:00          Neutral  \PYGZhy{}0.738  1998\PYGZhy{}07\PYGZhy{}06 17:00   

  Lowest ONI Mode Zone  ONI max  ONI min  
0         La Nina  GMT    \PYGZhy{}1.19    \PYGZhy{}0.93  
1         La Nina  GMT     0.55    \PYGZhy{}0.93  
2         Neutral  GMT    \PYGZhy{}0.11     0.17  
3         El Nino  GMT    \PYGZhy{}0.98     1.22  
4         Neutral  GMT    \PYGZhy{}0.41     0.01  
5         Neutral  GMT    \PYGZhy{}1.27     0.03  
6         Neutral  GMT    \PYGZhy{}1.17     0.35  
7         La Nina  GMT     0.31    \PYGZhy{}1.04  
8         Neutral  GMT    \PYGZhy{}0.41    \PYGZhy{}0.30  
9         La Nina  GMT     0.14    \PYGZhy{}0.78  
\end{sphinxVerbatim}

\end{sphinxuseclass}\end{sphinxVerbatimOutput}

\end{sphinxuseclass}

\chapter{Style the table}
\label{\detokenize{notebooks/regional_and_local/SL_Rankings_annual:style-the-table}}\label{\detokenize{notebooks/regional_and_local/SL_Rankings_annual:sl-rankings-results}}
\sphinxAtStartPar
Now we’ll add some style. The table will be colored by the ONI, if the event is classified as an El Nino or La Nina.

\begin{sphinxuseclass}{cell}
\begin{sphinxuseclass}{tag_hide-input}\begin{sphinxVerbatimOutput}

\begin{sphinxuseclass}{cell_output}
\begin{sphinxVerbatim}[commandchars=\\\{\}]
\PYGZlt{}pandas.io.formats.style.Styler at 0x151bd2f70\PYGZgt{}
\end{sphinxVerbatim}

\end{sphinxuseclass}\end{sphinxVerbatimOutput}

\end{sphinxuseclass}
\end{sphinxuseclass}

\part{Make Timeseries Plots}
\label{\detokenize{notebooks/regional_and_local/SL_Rankings_annual:make-timeseries-plots}}\label{\detokenize{notebooks/regional_and_local/SL_Rankings_annual:sl-rankings-timeseries}}
\sphinxAtStartPar
Now we’ll plot this data up in the time domain.


\chapter{Interactive Plots}
\label{\detokenize{notebooks/regional_and_local/SL_Rankings_annual:interactive-plots}}
\sphinxAtStartPar
This version is interactive. Bad for making pdf reports though.

\begin{sphinxuseclass}{cell}
\begin{sphinxuseclass}{tag_hide-input}
\end{sphinxuseclass}
\end{sphinxuseclass}
\begin{sphinxuseclass}{cell}\begin{sphinxVerbatimInput}

\begin{sphinxuseclass}{cell_input}
\begin{sphinxVerbatim}[commandchars=\\\{\}]
\PYG{n}{make\PYGZus{}plotlyFigure}\PYG{p}{(}\PYG{n}{rsl\PYGZus{}monthly\PYGZus{}mean}\PYG{p}{,} 
        \PYG{n}{rsl\PYGZus{}monthly\PYGZus{}max}\PYG{p}{,} \PYG{n}{rsl\PYGZus{}monthly\PYGZus{}min}\PYG{p}{,} 
        \PYG{n}{top\PYGZus{}10\PYGZus{}table}\PYG{p}{,} \PYG{n}{rsl\PYGZus{}subset}\PYG{p}{,}\PYG{l+m+mi}{2}\PYG{p}{)}
\end{sphinxVerbatim}

\end{sphinxuseclass}\end{sphinxVerbatimInput}
\begin{sphinxVerbatimOutput}

\begin{sphinxuseclass}{cell_output}
\end{sphinxuseclass}\end{sphinxVerbatimOutput}

\end{sphinxuseclass}

\chapter{Static Plots}
\label{\detokenize{notebooks/regional_and_local/SL_Rankings_annual:static-plots}}
\sphinxAtStartPar
Plotting them all so we can decide what to do with them. Click “show code cell output” to see all stations.

\begin{sphinxuseclass}{cell}
\begin{sphinxuseclass}{tag_hide-input}
\begin{sphinxuseclass}{tag_hide-output}
\end{sphinxuseclass}
\end{sphinxuseclass}
\end{sphinxuseclass}
\begin{figure}[htbp]
\centering
\capstart

\noindent\sphinxincludegraphics{{b827930c2a1f087092d409b27247bd84656d3504ee2990e12526b27172528ed7}.png}
\caption{This is a static version of the previous figure. I am using this static version because there is more control over the printed figure.}\label{\detokenize{notebooks/regional_and_local/SL_Rankings_annual:sl-rankings}}\end{figure}


\part{Create a map}
\label{\detokenize{notebooks/regional_and_local/SL_Rankings_annual:create-a-map}}\label{\detokenize{notebooks/regional_and_local/SL_Rankings_annual:sl-rankings-map}}
\sphinxAtStartPar
Now we’ll make a map of this information, using max and min at each station. First we need to organize our data.

\begin{sphinxuseclass}{cell}
\begin{sphinxuseclass}{tag_hide-input}
\end{sphinxuseclass}
\end{sphinxuseclass}
\sphinxAtStartPar
Next, we’ll plot it up.

\begin{sphinxuseclass}{cell}
\begin{sphinxuseclass}{tag_hide-output}
\begin{sphinxuseclass}{tag_hide-input}
\end{sphinxuseclass}
\end{sphinxuseclass}
\end{sphinxuseclass}
\begin{figure}[htbp]
\centering
\capstart

\noindent\sphinxincludegraphics{{cabdbab75cac7ec67da3415e4c6d22ddad712bf817f5be5b9026beeaa5d9552b}.png}
\caption{Lowest and highest observed sea levels, relative to MHHW.}\label{\detokenize{notebooks/regional_and_local/SL_Rankings_annual:id1}}\end{figure}







\renewcommand{\indexname}{Index}
\printindex
\end{document}